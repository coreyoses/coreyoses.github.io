\documentclass[11pt]{article}

\renewcommand{\rmdefault}{ppl}
\linespread{1.05}
\usepackage[scaled]{helvet}
\usepackage{courier}
\usepackage{euler}
\usepackage{soul}
%\usepackage{eulervm}
\normalfont
\usepackage[T1]{fontenc}

\usepackage[top=1in,left=1.25in,right=1.25in,bottom=1in]{geometry}
\usepackage[dvipsnames]{xcolor}
\usepackage{array}
\usepackage{hyperref}
\usepackage{enumitem}
\hypersetup{colorlinks,breaklinks,urlcolor=Maroon,linkcolor=Maroon}
\hyphenpenalty=10000
\setlength{\tabcolsep}{0pt}

\begin{document}
\noindent\hspace{0cm}\textcolor{Maroon}{\LARGE\textsc{\so{COREY OSES}}}

\noindent\hspace{0cm}\textit{Ph.D. Candidate in Materials Science,} Duke University

\vspace{0.5cm}

\noindent\hspace{0cm}\textcolor{black}{\textsc{\so{Personal Information}}}

\begin{center}
\begin{tabular}{m{1.0in}m{0.25in}c}
\raggedleft{\textit{\small{}}} & & 
\begin{tabular}{m{0.85in}m{0.15in}m{3.75in}}
\textit{\small{email}} & & \href{mailto:corey.oses@duke.edu}{corey.oses@duke.edu} \\ 
\end{tabular} \\ 
\end{tabular}
\end{center}

\vspace{-0.75cm}

\begin{center}
\begin{tabular}{m{1.0in}m{0.25in}c}
\raggedleft{\textit{\small{}}} & & 
\begin{tabular}{m{0.85in}m{0.15in}m{3.75in}}
\textit{\small{website}} & & \href{http://www.coreyoses.com}{http://www.coreyoses.com} \\ 
\end{tabular} \\ 
\end{tabular}
\end{center}

\vspace{-0.75cm}

\begin{center}
\begin{tabular}{m{1.0in}m{0.25in}c}
\raggedleft{\textit{\small{}}} & & 
\begin{tabular}{m{0.85in}m{0.15in}m{3.75in}}
\textit{\small{phone}} & & (M) +1 (201) 674 1407\ \ $\cdotp$\ \ (W) +1 (919) 684 1553 \\ 
\end{tabular} \\ 
\end{tabular}
\end{center}

\noindent\hspace{0cm}\textcolor{black}{\textsc{\so{Objective}}}

\begin{center}
\begin{tabular}{m{1.0in}m{0.25in}c}
\raggedleft{\textit{\small{}}} & & 
\begin{tabular}{m{0.85in}m{0.15in}m{3.75in}}
\multicolumn{3}{p{4.75in}}{\footnotesize{To obtain a graduate degree and perform research relevant to Materials Science and Engineering in order to develop my technical and managerial skills toward academia and make significant contributions to Duke University.}} 
\end{tabular} \\ 
\end{tabular}
\end{center}

\noindent\hspace{0cm}\textcolor{black}{\textsc{\so{Education}}}

\begin{center}
\begin{tabular}{m{1.0in}m{0.25in}c}
\raggedleft{\textit{\small{Doctor of Philosophy}}} & & 
\begin{tabular}{m{0.85in}m{0.15in}m{3.75in}}
\textit{\small{2013--Present}} & & Duke University \\ \multicolumn{3}{p{4.75in}}{\footnotesize{\mbox{GPA: 3.7/4.0}\ \ $\cdotp$\ \ \mbox{Department: Mechanical Engineering and Materials Science} \newline Thesis: \textit{Advanced Techniques in High-Throughput Computational Materials Science} \newline \textbf{Advisor}: Stefano Curtarolo}} 
\end{tabular} \\ 
\end{tabular}
\end{center}

\begin{center}
\begin{tabular}{m{1.0in}m{0.25in}c}
\raggedleft{\textit{\small{Bachelor of Science}}} & & 
\begin{tabular}{m{0.85in}m{0.15in}m{3.75in}}
\textit{\small{2009--2013}} & & Cornell University \\ \multicolumn{3}{p{4.75in}}{\footnotesize{\mbox{GPA: 3.3/4.0}\ \ $\cdotp$\ \ \mbox{Department: Applied and Engineering Physics} \newline Thesis: \textit{Plume Propagation Simulation for Pulsed Laser Deposition} \newline \textbf{Advisor}: Joel Brock}} 
\end{tabular} \\ 
\end{tabular}
\end{center}

\begin{center}
\begin{tabular}{m{1.0in}m{0.25in}c}
\raggedleft{\textit{\small{High School Diploma}}} & & 
\begin{tabular}{m{0.85in}m{0.15in}m{3.75in}}
\textit{\small{2005-2009}} & & Bloomfield High School \\ \multicolumn{3}{p{4.75in}}{\footnotesize{\mbox{GPA: 3.9/4.0}\ \ $\cdotp$\ \ \mbox{\textit{Graduated fifth in class of 428}}}} 
\end{tabular} \\ 
\end{tabular}
\end{center}

\noindent\hspace{0cm}\textcolor{black}{\textsc{\so{Research}}}

\begin{center}
\begin{tabular}{m{1.0in}m{0.25in}c}
\raggedleft{\textit{\small{Duke University}}} & & 
\begin{tabular}{m{0.85in}m{0.15in}m{3.75in}}
\textit{\small{2015--2018}} & & Advanced Techniques in High-Throughput Computational Materials Science \\ \multicolumn{3}{p{4.75in}}{\footnotesize{\textbf{Advisor}: Stefano Curtarolo}} 
\end{tabular} \\ 
\end{tabular}
\end{center}

\begin{center}
\begin{tabular}{m{1.0in}m{0.25in}c}
\raggedleft{\textit{\small{Duke University}}} & & 
\begin{tabular}{m{0.85in}m{0.15in}m{3.75in}}
\textit{\small{2014--2016}} & & Modeling Off-Stochiometry Materials \\ \multicolumn{3}{p{4.75in}}{\footnotesize{Developed and implemented a robust framework for modeling off-stoichiometry and aperiodic materials in a high-throughput fashion. \noindent\begin{itemize}[leftmargin=*] \item Presented at APS March Meeting 2016. \end{itemize} \textbf{Advisor}: Stefano Curtarolo}} 
\end{tabular} \\ 
\end{tabular}
\end{center}

\begin{center}
\begin{tabular}{m{1.0in}m{0.25in}c}
\raggedleft{\textit{\small{Duke University}}} & & 
\begin{tabular}{m{0.85in}m{0.15in}m{3.75in}}
\textit{\small{2014}} & & Materials Cartography \\ \multicolumn{3}{p{4.75in}}{\footnotesize{Developed novel fingerprinting method for electronic properties of materials that enabled the construction of similarity maps.\ \ $\cdotp$\ \ Collaborative effort between UNC-Chapel Hill and Duke University. \noindent\begin{itemize}[leftmargin=*] \item Presented at BYU Condensed Matter Physics Seminar --- February 18, 2016. \item Presented at Duke MEMS Department Graduate Student Seminar --- September 25, 2015. \item Presented at APS March Meeting 2015. \end{itemize} \textbf{Advisor}: Stefano Curtarolo}} 
\end{tabular} \\ 
\end{tabular}
\end{center}

\begin{center}
\begin{tabular}{m{1.0in}m{0.25in}c}
\raggedleft{\textit{\small{Cornell University}}} & & 
\begin{tabular}{m{0.85in}m{0.15in}m{3.75in}}
\textit{\small{Fall 2012--Spring 2013}} & & Plume Propagation Simulation for Pulsed Laser Deposition \\ \multicolumn{3}{p{4.75in}}{\footnotesize{Developed a robust, three-dimensional Monte-Carlo simulation of the Pulsed Laser Deposition material growth technique used at the Cornell High Energy Synchrotron Source. \noindent\begin{itemize}[leftmargin=*] \item Presented at NSF / AAAS / EHR Emerging Researchers National Conference 2014. \item Technical poster presentation, MRS / ASM / AVS / AReMS Meeting at NC State University --- November 15, 2013. \item \textcolor{NavyBlue}{Technical Poster and Paper Finalist}, SHPE Conference 2013. \item \textcolor{NavyBlue}{Best Presentation Award}, Duke MEMS Department Retreat 2013. \end{itemize} \textbf{Advisor}: Joel Brock}} 
\end{tabular} \\ 
\end{tabular}
\end{center}

\begin{center}
\begin{tabular}{m{1.0in}m{0.25in}c}
\raggedleft{\textit{\small{Cornell High Energy Synchrotron Source}}} & & 
\begin{tabular}{m{0.85in}m{0.15in}m{3.75in}}
\textit{\small{Summer 2012}} & & Synchrotron Radiation Focusing Optics --- Capillary Beam Stop Design \\ \multicolumn{3}{p{4.75in}}{\footnotesize{Designed and implemented a beam stop assembly to eliminate parasitic X-Ray beams and improve focusing capabilities of the ellipsoidal glass capillary optic. \noindent\begin{itemize}[leftmargin=*] \item \textcolor{NavyBlue}{First Place in Nanoscience and Physics Research Presentation}, NSF / AAAS / EHR Emerging Researchers National Conference 2013. \item Technical poster and research presentation, Cornell University LSAMP Research Symposium --- August 7, 2012. \end{itemize} \textbf{Advisors}: Ernest Fontes \& Rong Huang}} 
\end{tabular} \\ 
\end{tabular}
\end{center}

\begin{center}
\begin{tabular}{m{1.0in}m{0.25in}c}
\raggedleft{\textit{\small{Cornell University}}} & & 
\begin{tabular}{m{0.85in}m{0.15in}m{3.75in}}
\textit{\small{Fall 2011--Spring 2012}} & & Conductivity Behavior in Strontium Titanate \\ \multicolumn{3}{p{4.75in}}{\footnotesize{Developed and supported a model that characterizes the conductivity of annealed Strontium Titanate samples.\ \ $\cdotp$\ \ Further investigated conductivity behavior of annealed Strontium Titanate samples under varying electric potentials. \newline \textbf{Advisor}: Joel Brock}} 
\end{tabular} \\ 
\end{tabular}
\end{center}

\begin{center}
\begin{tabular}{m{1.0in}m{0.25in}c}
\raggedleft{\textit{\small{Cornell University}}} & & 
\begin{tabular}{m{0.85in}m{0.15in}m{3.75in}}
\textit{\small{2009--2011}} & & Cornell University Autonomous Flight Team \\ \multicolumn{3}{p{4.75in}}{\footnotesize{Constructed an autonomous plane with capabilities to navigate waypoints, survey areas, and retrieve visual information about the surfaces below as part of a team effort for AUVSI's (Association for Unmanned Vehicle Systems International) Student Unmanned Air Systems Competition. \noindent\begin{itemize}[leftmargin=*] \item Served as \textcolor{NavyBlue}{team's safety officer} and \textcolor{NavyBlue}{head system manager}, AUVSI Student Unmanned Air System (SUAS) 2010 Competition. \item \textcolor{NavyBlue}{Won a \$1,000 grant}, AUVSI Student Unmanned Air System (SUAS) 2010 Competition. \end{itemize} \textbf{Advisor}: Ashutosh Saxena}} 
\end{tabular} \\ 
\end{tabular}
\end{center}

\begin{center}
\begin{tabular}{m{1.0in}m{0.25in}c}
\raggedleft{\textit{\small{Cornell University}}} & & 
\begin{tabular}{m{0.85in}m{0.15in}m{3.75in}}
\textit{\small{2009--2010}} & & Meinig Family Cornell National Scholars \\ \multicolumn{3}{p{4.75in}}{\footnotesize{Collaborated with MFCNS, scholarship fund director, and the Cornell Alumni Association for the 2009--2010 annual research project, Academic Integrity, culminating with group presentation and discussion with relevant Cornell faculty and professors. \newline \textbf{Advisor}: Kristine M. DeLuca}} 
\end{tabular} \\ 
\end{tabular}
\end{center}

\noindent\hspace{0cm}\textcolor{black}{\textsc{\so{Teaching Experience}}}

\begin{center}
\begin{tabular}{m{1.0in}m{0.25in}c}
\raggedleft{\textit{\small{Teaching Assistant}}} & & 
\begin{tabular}{m{0.85in}m{0.15in}m{3.75in}}
\textit{\small{Fall 2014--Spring 2015}} & & ME 221:  Structure and Properties of Solids, Duke University \\ \multicolumn{3}{p{4.75in}}{\footnotesize{Introduction to materials science and engineering, emphasizing the relationships between the structure of a solid and its properties. Atomic and molecular origins of electrical, mechanical, and chemical behavior are treated in some detail for metals, alloys, polymers, ceramics, glasses, and composite materials. \noindent\begin{itemize}[leftmargin=*] \item \textcolor{NavyBlue}{Best Teaching Assistant Award}, Spring 2015 \vspace*{-\baselineskip} \end{itemize}}} 
\end{tabular} \\ 
\end{tabular}
\end{center}

\noindent\hspace{0cm}\textcolor{black}{\textsc{\so{Work Experience and Skills}}}

\begin{center}
\begin{tabular}{m{1.0in}m{0.25in}c}
\raggedleft{\textit{\small{Graduate}}} & & 
\begin{tabular}{m{0.85in}m{0.15in}m{3.75in}}
\textit{\small{January 2015}} & & Machine Learning Summer School at the University of Texas, Austin \\ 
\end{tabular} \\ 
\end{tabular}
\end{center}

\vspace{-0.75cm}

\begin{center}
\begin{tabular}{m{1.0in}m{0.25in}c}
\raggedleft{\textit{\small{Internship}}} & & 
\begin{tabular}{m{0.85in}m{0.15in}m{3.75in}}
\textit{\small{Summer 2013}} & & Cornell High Energy Sychrotron Source (BioSAXS on F2 and G Beamlines) \\ \multicolumn{3}{p{4.75in}}{\footnotesize{\textbf{Supervisors}: Richard Edward Gillilan \& Ernest Fontes}} 
\end{tabular} \\ 
\end{tabular}
\end{center}

\vspace{-0.75cm}

\begin{center}
\begin{tabular}{m{1.0in}m{0.25in}c}
\raggedleft{\textit{\small{Graduate}}} & & 
\begin{tabular}{m{0.85in}m{0.15in}m{3.75in}}
\textit{\small{May 2011}} & & \textit{The LeaderShape Institute} \\ 
\end{tabular} \\ 
\end{tabular}
\end{center}

\vspace{-0.75cm}

\begin{center}
\begin{tabular}{m{1.0in}m{0.25in}c}
\raggedleft{\textit{\small{Student Employee}}} & & 
\begin{tabular}{m{0.85in}m{0.15in}m{3.75in}}
\textit{\small{Summer 2011}} & & ILR Budget Office, Cornell University \\ \multicolumn{3}{p{4.75in}}{\footnotesize{\textbf{Supervisor}: Renee Laree Monroe}} 
\end{tabular} \\ 
\end{tabular}
\end{center}

\vspace{-0.75cm}

\begin{center}
\begin{tabular}{m{1.0in}m{0.25in}c}
\raggedleft{\textit{\small{Technician License}}} & & 
\begin{tabular}{m{0.85in}m{0.15in}m{3.75in}}
\textit{\small{July 2010}} & & American Radio Relay League (ARRL) \\ 
\end{tabular} \\ 
\end{tabular}
\end{center}

\vspace{-0.75cm}

\begin{center}
\begin{tabular}{m{1.0in}m{0.25in}c}
\raggedleft{\textit{\small{Internship}}} & & 
\begin{tabular}{m{0.85in}m{0.15in}m{3.75in}}
\textit{\small{March 2010}} & & Supreme Court of New York \\ \multicolumn{3}{p{4.75in}}{\footnotesize{\textbf{Supervisors}: Ariel E. Belen \& Allen Hurkin-Torres}} 
\end{tabular} \\ 
\end{tabular}
\end{center}

\vspace{-0.75cm}

\begin{center}
\begin{tabular}{m{1.0in}m{0.25in}c}
\raggedleft{\textit{\small{Math Tutor}}} & & 
\begin{tabular}{m{0.85in}m{0.15in}m{3.75in}}
\textit{\small{Fall 2008}} & & Graduate Record Exam (GRE) \\ 
\end{tabular} \\ 
\end{tabular}
\end{center}

\vspace{-0.75cm}

\begin{center}
\begin{tabular}{m{1.0in}m{0.25in}c}
\raggedleft{\textit{\small{Office Assistant}}} & & 
\begin{tabular}{m{0.85in}m{0.15in}m{3.75in}}
\textit{\small{Summer 2008}} & & SOS Security, LLC in Parsippany, NJ \\ \multicolumn{3}{p{4.75in}}{\footnotesize{\textbf{Supervisor}: James Flanagan}} 
\end{tabular} \\ 
\end{tabular}
\end{center}

\vspace{-0.75cm}

\begin{center}
\begin{tabular}{m{1.0in}m{0.25in}c}
\raggedleft{\textit{\small{Proficient Coder}}} & & 
\begin{tabular}{m{0.85in}m{0.15in}m{3.75in}}
\textit{\small{Present}} & & \texttt{Python}, \LaTeX, \texttt{C++}, \texttt{Matlab}, \& \texttt{R} \\ 
\end{tabular} \\ 
\end{tabular}
\end{center}

\noindent\hspace{0cm}\textcolor{black}{\textsc{\so{Activities and Outreach}}}

\begin{center}
\begin{tabular}{m{1.0in}m{0.25in}c}
\raggedleft{\textit{\small{Member}}} & & 
\begin{tabular}{m{0.85in}m{0.15in}m{3.75in}}
\textit{\small{2014--Present}} & & American Physical Society \\ 
\end{tabular} \\ 
\end{tabular}
\end{center}

\vspace{-0.75cm}

\begin{center}
\begin{tabular}{m{1.0in}m{0.25in}c}
\raggedleft{\textit{\small{Graduate Student Advisor}}} & & 
\begin{tabular}{m{0.85in}m{0.15in}m{3.75in}}
\textit{\small{2009--Present}} & & Society of Hispanic Professional Engineers \\ \multicolumn{3}{p{4.75in}}{\footnotesize{\textbf{Positions}: Graduate Student Advisor, \textcolor{NavyBlue}{President}, Corporate Vice President \& Treasurer}} 
\end{tabular} \\ 
\end{tabular}
\end{center}

\vspace{-0.75cm}

\begin{center}
\begin{tabular}{m{1.0in}m{0.25in}c}
\raggedleft{\textit{\small{\textcolor{NavyBlue}{Distinguished Past Governor}}}} & & 
\begin{tabular}{m{0.85in}m{0.15in}m{3.75in}}
\textit{\small{2009--2013}} & & Circle K, International \\ \multicolumn{3}{p{4.75in}}{\footnotesize{\textbf{Positions}: New York District \textcolor{NavyBlue}{Distinguished Past Governor}, New York District \textcolor{NavyBlue}{Distinguished Past Treasurer} \& \textcolor{NavyBlue}{Restarting Chapter President} at Cornell University}} 
\end{tabular} \\ 
\end{tabular}
\end{center}

\vspace{-0.75cm}

\begin{center}
\begin{tabular}{m{1.0in}m{0.25in}c}
\raggedleft{\textit{\small{Member}}} & & 
\begin{tabular}{m{0.85in}m{0.15in}m{3.75in}}
\textit{\small{2009--2013}} & & Meinig Family Cornell National Scholars, Cornell Commitment \\ 
\end{tabular} \\ 
\end{tabular}
\end{center}

\vspace{-0.75cm}

\begin{center}
\begin{tabular}{m{1.0in}m{0.25in}c}
\raggedleft{\textit{\small{Member}}} & & 
\begin{tabular}{m{0.85in}m{0.15in}m{3.75in}}
\textit{\small{2009}} & & United Astronomy Clubs of New Jersey \\ 
\end{tabular} \\ 
\end{tabular}
\end{center}

\vspace{-0.75cm}

\begin{center}
\begin{tabular}{m{1.0in}m{0.25in}c}
\raggedleft{\textit{\small{Member}}} & & 
\begin{tabular}{m{0.85in}m{0.15in}m{3.75in}}
\textit{\small{2009}} & & New Jersey Astronomical Group \\ 
\end{tabular} \\ 
\end{tabular}
\end{center}

\vspace{-0.75cm}

\begin{center}
\begin{tabular}{m{1.0in}m{0.25in}c}
\raggedleft{\textit{\small{\textcolor{NavyBlue}{President}}}} & & 
\begin{tabular}{m{0.85in}m{0.15in}m{3.75in}}
\textit{\small{2008--2009}} & & Astronomy Club \\ 
\end{tabular} \\ 
\end{tabular}
\end{center}

\vspace{-0.75cm}

\begin{center}
\begin{tabular}{m{1.0in}m{0.25in}c}
\raggedleft{\textit{\small{\textcolor{NavyBlue}{President}}}} & & 
\begin{tabular}{m{0.85in}m{0.15in}m{3.75in}}
\textit{\small{2005--2009}} & & Future Business Leaders of America \\ \multicolumn{3}{p{4.75in}}{\footnotesize{\textbf{Positions}: \textcolor{NavyBlue}{President} \& General Manager of School Store}} 
\end{tabular} \\ 
\end{tabular}
\end{center}

\vspace{-0.75cm}

\begin{center}
\begin{tabular}{m{1.0in}m{0.25in}c}
\raggedleft{\textit{\small{Membership Director}}} & & 
\begin{tabular}{m{0.85in}m{0.15in}m{3.75in}}
\textit{\small{2005--2009}} & & Key Club \\ \multicolumn{3}{p{4.75in}}{\footnotesize{\textbf{Positions}: Membership Director \& Activities Direcotr}} 
\end{tabular} \\ 
\end{tabular}
\end{center}

\vspace{-0.75cm}

\begin{center}
\begin{tabular}{m{1.0in}m{0.25in}c}
\raggedleft{\textit{\small{Retreat Team}}} & & 
\begin{tabular}{m{0.85in}m{0.15in}m{3.75in}}
\textit{\small{2005--2009}} & & Saint Thomas the Apostle Youth Group \\ \multicolumn{3}{p{4.75in}}{\footnotesize{\textbf{Positions}: Retreat Team, Lead Role in \textit{Stations of the Cross Performance} \& Confirmation Class Instructor}} 
\end{tabular} \\ 
\end{tabular}
\end{center}

\vspace{-0.75cm}

\begin{center}
\begin{tabular}{m{1.0in}m{0.25in}c}
\raggedleft{\textit{\small{Secretary}}} & & 
\begin{tabular}{m{0.85in}m{0.15in}m{3.75in}}
\textit{\small{2006--2009}} & & Model United Nations \\ \multicolumn{3}{p{4.75in}}{\footnotesize{\textbf{Positions}: Secretary \& Treasurer}} 
\end{tabular} \\ 
\end{tabular}
\end{center}

\vspace{-0.75cm}

\begin{center}
\begin{tabular}{m{1.0in}m{0.25in}c}
\raggedleft{\textit{\small{Treasurer}}} & & 
\begin{tabular}{m{0.85in}m{0.15in}m{3.75in}}
\textit{\small{2006--2009}} & & Math Team \\ 
\end{tabular} \\ 
\end{tabular}
\end{center}

\vspace{-0.75cm}

\begin{center}
\begin{tabular}{m{1.0in}m{0.25in}c}
\raggedleft{\textit{\small{Member}}} & & 
\begin{tabular}{m{0.85in}m{0.15in}m{3.75in}}
\textit{\small{2006--2009}} & & Science Club \\ \multicolumn{3}{p{4.75in}}{\footnotesize{Physics Club\ \ $\cdotp$\ \ Chemistry Club}} 
\end{tabular} \\ 
\end{tabular}
\end{center}

\vspace{-0.75cm}

\begin{center}
\begin{tabular}{m{1.0in}m{0.25in}c}
\raggedleft{\textit{\small{Athlete}}} & & 
\begin{tabular}{m{0.85in}m{0.15in}m{3.75in}}
\textit{\small{2006--2009}} & & Spring Track and Field \\ \multicolumn{3}{p{4.75in}}{\footnotesize{Javelin Junior Varsity Team}} 
\end{tabular} \\ 
\end{tabular}
\end{center}

\vspace{-0.75cm}

\begin{center}
\begin{tabular}{m{1.0in}m{0.25in}c}
\raggedleft{\textit{\small{Member}}} & & 
\begin{tabular}{m{0.85in}m{0.15in}m{3.75in}}
\textit{\small{2006--2009}} & & Weight Lifting Team \\ 
\end{tabular} \\ 
\end{tabular}
\end{center}

\vspace{-0.75cm}

\begin{center}
\begin{tabular}{m{1.0in}m{0.25in}c}
\raggedleft{\textit{\small{Tutor}}} & & 
\begin{tabular}{m{0.85in}m{0.15in}m{3.75in}}
\textit{\small{2006--2009}} & & Tutor at the Library \\ 
\end{tabular} \\ 
\end{tabular}
\end{center}

\vspace{-0.75cm}

\begin{center}
\begin{tabular}{m{1.0in}m{0.25in}c}
\raggedleft{\textit{\small{Member}}} & & 
\begin{tabular}{m{0.85in}m{0.15in}m{3.75in}}
\textit{\small{2005--2008}} & & Latin Club \\ 
\end{tabular} \\ 
\end{tabular}
\end{center}

\vspace{-0.75cm}

\begin{center}
\begin{tabular}{m{1.0in}m{0.25in}c}
\raggedleft{\textit{\small{Member}}} & & 
\begin{tabular}{m{0.85in}m{0.15in}m{3.75in}}
\textit{\small{2005--2006}} & & Bowling Team \\ 
\end{tabular} \\ 
\end{tabular}
\end{center}

\vspace{-0.75cm}

\begin{center}
\begin{tabular}{m{1.0in}m{0.25in}c}
\raggedleft{\textit{\small{Black belt}}} & & 
\begin{tabular}{m{0.85in}m{0.15in}m{3.75in}}
\textit{\small{2002--2006}} & & Tae Kwon Do \\ 
\end{tabular} \\ 
\end{tabular}
\end{center}

\noindent\hspace{0cm}\textcolor{black}{\textsc{\so{Press and News Releases}}}

\begin{center}
\begin{tabular}{m{1.0in}m{0.25in}c}
\raggedleft{\textit{\small{Computational Chemistry Highlights}}} & & 
\begin{tabular}{m{0.85in}m{0.15in}m{3.75in}}
\textit{\small{January 2015}} & & \textit{``Materials Cartography: Representing and Mining Materials Space Using Structural and Electronic Fingerprints''} \\ \multicolumn{3}{p{4.75in}}{\footnotesize{``This paper is a \textcolor{NavyBlue}{\textit{tour de force}} for computational materials science'' --- Prof. Al\'{a}n Aspuru-Guzik, Harvard University. \newline \href{http://www.compchemhighlights.org/2015/01/materials-cartography-representing-and.html}{http://www.compchemhighlights.org/2015/01/materials-cartography-representing-and.html}}} 
\end{tabular} \\ 
\end{tabular}
\end{center}

\vspace{-0.75cm}

\begin{center}
\begin{tabular}{m{1.0in}m{0.25in}c}
\raggedleft{\textit{\small{Duke University Research}}} & & 
\begin{tabular}{m{0.85in}m{0.15in}m{3.75in}}
\textit{\small{January 2015}} & & \textit{``Molecular Tornado''} \\ \multicolumn{3}{p{4.75in}}{\footnotesize{\href{https://research.duke.edu/molecular-tornado}{https://research.duke.edu/molecular-tornado}}} 
\end{tabular} \\ 
\end{tabular}
\end{center}

\vspace{-0.75cm}

\begin{center}
\begin{tabular}{m{1.0in}m{0.25in}c}
\raggedleft{\textit{\small{Duke University}}} & & 
\begin{tabular}{m{0.85in}m{0.15in}m{3.75in}}
\textit{\small{June 2014}} & & \textit{``Pratt Profiles: Corey Oses''} \\ \multicolumn{3}{p{4.75in}}{\footnotesize{\href{http://pratt.duke.edu/graduate/diversity/pratt-profiles-corey-oses}{http://pratt.duke.edu/graduate/diversity/pratt-profiles-corey-oses}}} 
\end{tabular} \\ 
\end{tabular}
\end{center}

\vspace{-0.75cm}

\begin{center}
\begin{tabular}{m{1.0in}m{0.25in}c}
\raggedleft{\textit{\small{New York Kiwanis}}} & & 
\begin{tabular}{m{0.85in}m{0.15in}m{3.75in}}
\textit{\small{February 2013}} & & \textit{``New York Kiwanis Mid-Winter Conference 2013''} \\ \multicolumn{3}{p{4.75in}}{\footnotesize{\href{http://www.kiwanis-ny.org/1213/midyear.htm}{http://www.kiwanis-ny.org/1213/midyear.htm}}} 
\end{tabular} \\ 
\end{tabular}
\end{center}

\vspace{-0.75cm}

\begin{center}
\begin{tabular}{m{1.0in}m{0.25in}c}
\raggedleft{\textit{\small{New York Kiwanis}}} & & 
\begin{tabular}{m{0.85in}m{0.15in}m{3.75in}}
\textit{\small{June 2012}} & & \textit{``K-Kids Show Talent for Fundraising''} \\ \multicolumn{3}{p{4.75in}}{\footnotesize{\href{http://www.kiwanis-ny.org/1213/midyear.htm}{http://www.kiwanis-ny.org/1213/midyear.htm}}} 
\end{tabular} \\ 
\end{tabular}
\end{center}

\vspace{-0.75cm}

\begin{center}
\begin{tabular}{m{1.0in}m{0.25in}c}
\raggedleft{\textit{\small{New York Kiwanis}}} & & 
\begin{tabular}{m{0.85in}m{0.15in}m{3.75in}}
\textit{\small{March 2012}} & & \textit{``Past Circle K Governors Help Celebrate 50th Convention''} \\ \multicolumn{3}{p{4.75in}}{\footnotesize{Elected Governor of New York Circle K. \newline \href{http://www.kiwanis-ny.org/news/view\_news.php?nid=618}{http://www.kiwanis-ny.org/news/view\_news.php?nid=618}}} 
\end{tabular} \\ 
\end{tabular}
\end{center}

\vspace{-0.75cm}

\begin{center}
\begin{tabular}{m{1.0in}m{0.25in}c}
\raggedleft{\textit{\small{Cornell University}}} & & 
\begin{tabular}{m{0.85in}m{0.15in}m{3.75in}}
\textit{\small{March 2011}} & & \textit{``Undergraduate Student of the Month''} \\ \multicolumn{3}{p{4.75in}}{\footnotesize{\href{https://www.engineering.cornell.edu/diversity/about/honors/students/2011-03.cfm}{https://www.engineering.cornell.edu/diversity/about/honors/students/2011-03.cfm}}} 
\end{tabular} \\ 
\end{tabular}
\end{center}

\noindent\hspace{0cm}\textcolor{black}{\textsc{\so{Honors and Awards}}}

\begin{center}
\begin{tabular}{m{1.0in}m{0.25in}c}
\raggedleft{\textit{\small{Fellowship}}} & & 
\begin{tabular}{m{0.85in}m{0.15in}m{3.75in}}
\textit{\small{2013-2016}} & & NSF Graduate Research Fellowship, National Science Foundation \\ 
\end{tabular} \\ 
\end{tabular}
\end{center}

\vspace{-0.75cm}

\begin{center}
\begin{tabular}{m{1.0in}m{0.25in}c}
\raggedleft{\textit{\small{Fellowship}}} & & 
\begin{tabular}{m{0.85in}m{0.15in}m{3.75in}}
\textit{\small{2013-2015}} & & GEM Associate Fellowship, The National GEM Consortium \\ 
\end{tabular} \\ 
\end{tabular}
\end{center}

\vspace{-0.75cm}

\begin{center}
\begin{tabular}{m{1.0in}m{0.25in}c}
\raggedleft{\textit{\small{City Citation}}} & & 
\begin{tabular}{m{0.85in}m{0.15in}m{3.75in}}
\textit{\small{March 21, 2013}} & & New York City Citation as Circle K Governor, Council Member Fernando Cabrera \\ 
\end{tabular} \\ 
\end{tabular}
\end{center}

\vspace{-0.75cm}

\begin{center}
\begin{tabular}{m{1.0in}m{0.25in}c}
\raggedleft{\textit{\small{Award}}} & & 
\begin{tabular}{m{0.85in}m{0.15in}m{3.75in}}
\textit{\small{October 8, 2011}} & & HENAAC College Bowl Winner \\ \multicolumn{3}{p{4.75in}}{\footnotesize{Northrop Grumman One Team}} 
\end{tabular} \\ 
\end{tabular}
\end{center}

\vspace{-0.75cm}

\begin{center}
\begin{tabular}{m{1.0in}m{0.25in}c}
\raggedleft{\textit{\small{Scholarship}}} & & 
\begin{tabular}{m{0.85in}m{0.15in}m{3.75in}}
\textit{\small{2011--2013}} & & Shell Incentive Fund Scholarship \\ 
\end{tabular} \\ 
\end{tabular}
\end{center}

\vspace{-0.75cm}

\begin{center}
\begin{tabular}{m{1.0in}m{0.25in}c}
\raggedleft{\textit{\small{Honor}}} & & 
\begin{tabular}{m{0.85in}m{0.15in}m{3.75in}}
\textit{\small{2010--2013}} & & Louis Stokes Alliance for Minority Participation (LSAMP) Scholar \\ 
\end{tabular} \\ 
\end{tabular}
\end{center}

\vspace{-0.75cm}

\begin{center}
\begin{tabular}{m{1.0in}m{0.25in}c}
\raggedleft{\textit{\small{Scholarship}}} & & 
\begin{tabular}{m{0.85in}m{0.15in}m{3.75in}}
\textit{\small{2010 \& 2011}} & & Xerox Corporation Scholarship \\ 
\end{tabular} \\ 
\end{tabular}
\end{center}

\vspace{-0.75cm}

\begin{center}
\begin{tabular}{m{1.0in}m{0.25in}c}
\raggedleft{\textit{\small{Scholarship}}} & & 
\begin{tabular}{m{0.85in}m{0.15in}m{3.75in}}
\textit{\small{2010 \& 2011}} & & Intel Academic Award \\ 
\end{tabular} \\ 
\end{tabular}
\end{center}

\vspace{-0.75cm}

\begin{center}
\begin{tabular}{m{1.0in}m{0.25in}c}
\raggedleft{\textit{\small{Scholarship}}} & & 
\begin{tabular}{m{0.85in}m{0.15in}m{3.75in}}
\textit{\small{2010--2013}} & & GE Foundation / LULAC Scholarship \\ 
\end{tabular} \\ 
\end{tabular}
\end{center}

\vspace{-0.75cm}

\begin{center}
\begin{tabular}{m{1.0in}m{0.25in}c}
\raggedleft{\textit{\small{Scholarship}}} & & 
\begin{tabular}{m{0.85in}m{0.15in}m{3.75in}}
\textit{\small{2009--2013}} & & Meinig Family Cornell National Scholars, Awarded by Peter Meinig (Past Chairman of the Board of Trustees at Cornell University) \\ 
\end{tabular} \\ 
\end{tabular}
\end{center}

\vspace{-0.75cm}

\begin{center}
\begin{tabular}{m{1.0in}m{0.25in}c}
\raggedleft{\textit{\small{Scholarship}}} & & 
\begin{tabular}{m{0.85in}m{0.15in}m{3.75in}}
\textit{\small{2009}} & & \textit{Men of Principle} Award, Beta Delta Chapter of Beta Theta Pi \\ 
\end{tabular} \\ 
\end{tabular}
\end{center}

\vspace{-0.75cm}

\begin{center}
\begin{tabular}{m{1.0in}m{0.25in}c}
\raggedleft{\textit{\small{Scholarship}}} & & 
\begin{tabular}{m{0.85in}m{0.15in}m{3.75in}}
\textit{\small{2009}} & & \textcolor{NavyBlue}{Gold Medallion Winner in Engineering and Mathematics}, Hispanic Heritage Youth Awards \\ 
\end{tabular} \\ 
\end{tabular}
\end{center}

\vspace{-0.75cm}

\begin{center}
\begin{tabular}{m{1.0in}m{0.25in}c}
\raggedleft{\textit{\small{Scholarship}}} & & 
\begin{tabular}{m{0.85in}m{0.15in}m{3.75in}}
\textit{\small{2009}} & & New Jersey Principals and Supervisors Association Scholarship \\ 
\end{tabular} \\ 
\end{tabular}
\end{center}

\vspace{-0.75cm}

\begin{center}
\begin{tabular}{m{1.0in}m{0.25in}c}
\raggedleft{\textit{\small{Scholarship}}} & & 
\begin{tabular}{m{0.85in}m{0.15in}m{3.75in}}
\textit{\small{2009}} & & Edward J. Bloustein Distinguished Scholar \\ 
\end{tabular} \\ 
\end{tabular}
\end{center}

\vspace{-0.75cm}

\begin{center}
\begin{tabular}{m{1.0in}m{0.25in}c}
\raggedleft{\textit{\small{Scholarship}}} & & 
\begin{tabular}{m{0.85in}m{0.15in}m{3.75in}}
\textit{\small{2009}} & & Investors Savings Bank Scholarship \\ 
\end{tabular} \\ 
\end{tabular}
\end{center}

\vspace{-0.75cm}

\begin{center}
\begin{tabular}{m{1.0in}m{0.25in}c}
\raggedleft{\textit{\small{Scholarship}}} & & 
\begin{tabular}{m{0.85in}m{0.15in}m{3.75in}}
\textit{\small{2009}} & & Bloomfield Education Association Scholarship \\ 
\end{tabular} \\ 
\end{tabular}
\end{center}

\vspace{-0.75cm}

\begin{center}
\begin{tabular}{m{1.0in}m{0.25in}c}
\raggedleft{\textit{\small{Scholarship}}} & & 
\begin{tabular}{m{0.85in}m{0.15in}m{3.75in}}
\textit{\small{2009}} & & Special Recognition Award, Bloomfield Kiwanis \\ 
\end{tabular} \\ 
\end{tabular}
\end{center}

\vspace{-0.75cm}

\begin{center}
\begin{tabular}{m{1.0in}m{0.25in}c}
\raggedleft{\textit{\small{Scholarship}}} & & 
\begin{tabular}{m{0.85in}m{0.15in}m{3.75in}}
\textit{\small{2009}} & & The Harold Brotherhood Award \\ 
\end{tabular} \\ 
\end{tabular}
\end{center}

\vspace{-0.75cm}

\begin{center}
\begin{tabular}{m{1.0in}m{0.25in}c}
\raggedleft{\textit{\small{Scholarship}}} & & 
\begin{tabular}{m{0.85in}m{0.15in}m{3.75in}}
\textit{\small{2009}} & & Jean Doswell Oakes Scholarship, Oakeside Cultural Center \\ 
\end{tabular} \\ 
\end{tabular}
\end{center}

\vspace{-0.75cm}

\begin{center}
\begin{tabular}{m{1.0in}m{0.25in}c}
\raggedleft{\textit{\small{Scholarship}}} & & 
\begin{tabular}{m{0.85in}m{0.15in}m{3.75in}}
\textit{\small{2009}} & & \textcolor{NavyBlue}{Superintendent's Bengal Pride Award} for Excellence in Academics and Citizenship \\ 
\end{tabular} \\ 
\end{tabular}
\end{center}

\vspace{-0.75cm}

\begin{center}
\begin{tabular}{m{1.0in}m{0.25in}c}
\raggedleft{\textit{\small{Award}}} & & 
\begin{tabular}{m{0.85in}m{0.15in}m{3.75in}}
\textit{\small{2009}} & & \textcolor{NavyBlue}{Outstanding Student Citizen} for Youth Week, Bloomfield High School \\ 
\end{tabular} \\ 
\end{tabular}
\end{center}

\vspace{-0.75cm}

\begin{center}
\begin{tabular}{m{1.0in}m{0.25in}c}
\raggedleft{\textit{\small{Award}}} & & 
\begin{tabular}{m{0.85in}m{0.15in}m{3.75in}}
\textit{\small{2009}} & & \textcolor{NavyBlue}{First Place Impromptu Essay}, New Jersey District Key Club Convention \\ 
\end{tabular} \\ 
\end{tabular}
\end{center}

\vspace{-0.75cm}

\begin{center}
\begin{tabular}{m{1.0in}m{0.25in}c}
\raggedleft{\textit{\small{Scholarship}}} & & 
\begin{tabular}{m{0.85in}m{0.15in}m{3.75in}}
\textit{\small{2009}} & & Good Citizen Award, The Daughters of the American Revolution \\ 
\end{tabular} \\ 
\end{tabular}
\end{center}

\vspace{-0.75cm}

\begin{center}
\begin{tabular}{m{1.0in}m{0.25in}c}
\raggedleft{\textit{\small{Honor}}} & & 
\begin{tabular}{m{0.85in}m{0.15in}m{3.75in}}
\textit{\small{2008--2009}} & & National Honor Society \\ 
\end{tabular} \\ 
\end{tabular}
\end{center}

\vspace{-0.75cm}

\begin{center}
\begin{tabular}{m{1.0in}m{0.25in}c}
\raggedleft{\textit{\small{Honor}}} & & 
\begin{tabular}{m{0.85in}m{0.15in}m{3.75in}}
\textit{\small{2008--2009}} & & Bloomfield High School Scholar Athlete \\ 
\end{tabular} \\ 
\end{tabular}
\end{center}

\vspace{-0.75cm}

\begin{center}
\begin{tabular}{m{1.0in}m{0.25in}c}
\raggedleft{\textit{\small{Honor}}} & & 
\begin{tabular}{m{0.85in}m{0.15in}m{3.75in}}
\textit{\small{2006--2009}} & & Bloomfield High School Academic Excellence \\ 
\end{tabular} \\ 
\end{tabular}
\end{center}

\vspace{-0.75cm}

\begin{center}
\begin{tabular}{m{1.0in}m{0.25in}c}
\raggedleft{\textit{\small{Honor}}} & & 
\begin{tabular}{m{0.85in}m{0.15in}m{3.75in}}
\textit{\small{2008}} & & National Ventures Scholar Program \\ 
\end{tabular} \\ 
\end{tabular}
\end{center}

\vspace{-0.75cm}

\begin{center}
\begin{tabular}{m{1.0in}m{0.25in}c}
\raggedleft{\textit{\small{Award}}} & & 
\begin{tabular}{m{0.85in}m{0.15in}m{3.75in}}
\textit{\small{2008}} & & Yale Book Club Award, Yale Club of Montclair \\ 
\end{tabular} \\ 
\end{tabular}
\end{center}

\noindent\hspace{0cm}\textcolor{black}{\textsc{\so{Publications}}}

\vspace{0.5cm}

\noindent\hspace{0cm}{\Large2016}

\begin{center}
\begin{tabular}{m{1.0in}m{0.25in}c}
\raggedleft{\textit{\small{In Preparation}}} & & 
\begin{tabular}{m{0.85in}m{0.15in}m{3.75in}}
\multicolumn{3}{p{4.75in}}{\textit{The statistics of 6A non-metal binary and ternary structures --- what can we learn from nature?}} \\ \multicolumn{3}{p{4.75in}}{\footnotesize{\textbf{Authors}: Alon Hever, \textcolor{NavyBlue}{Corey Oses}, Stefano Curtarolo, Ohad Levy \& Amir Natan \newline \textbf{Abstract}: The fundamental principles underlying the arrangement of the elements into solid compounds with an enormous variety of crystal structures are still largely unknown. In this study we obtain a general overview of the structure types appearing in an important subset of the solid compounds, i.e. binary and ternary compounds of the 6A column non-metals, oxides, sulfides and selenides. We present an analysis of these compounds, including the prevalence of various structure types, their symmetry properties, compositions, stoichiometries and unit cell sizes. We demonstrate that there are preferred stoichiometries and structure types. We further show that some structure types and stoichiometries have highly specific atomic compositions which may reflect both the chemistry and research bias.}} 
\end{tabular} \\ 
\end{tabular}
\end{center}

\begin{center}
\begin{tabular}{m{1.0in}m{0.25in}c}
\raggedleft{\textit{\small{Submitted}}} & & 
\begin{tabular}{m{0.85in}m{0.15in}m{3.75in}}
\multicolumn{3}{p{4.75in}}{\textit{Universal Fragment Descriptors for Predicting Electronic Properties of Inorganic Crystals}} \\ \multicolumn{3}{p{4.75in}}{\footnotesize{\textbf{Authors}: Olexandr Isayev, \textcolor{NavyBlue}{Corey Oses}, Stefano Curtarolo \& Alexander Tropsha \newline \textbf{Abstract}: Historically, materials discovery is driven by a laborious trial-and-error process. The growth of materials databases and emerging informatics approaches finally offer the opportunity to transform this practice into data- and knowledge-driven rational design-accelerating discovery of novel materials exhibiting desired properties. By using data from the \textsc{Aflow} repository for high-throughput ab-initio calculations, we have generated \underline{Q}uantitative \underline{M}aterials \underline{S}tructure-\underline{P}roperty \underline{R}elationship (QMSPR) models to predict three critical material properties, namely the metal/insulator classification, Fermi energy, and band gap energy. The prediction accuracy obtained with these QMSPR models approaches training data for virtually any stoichiometric inorganic crystalline material. We attribute the success and universality of these models to the construction of new material descriptors-referred to as the universal \underline{p}roperty-\underline{l}abeled \underline{m}aterial \underline{f}ragments (PLMF). This representation affords straightforward model interpretation in terms of simple heuristic design rules that could guide rational materials design. This proof-of-concept study demonstrates the power of materials informatics to dramatically accelerate the search for new materials. \newline \textbf{arXiv}: \href{http://arxiv.org/abs/1608.04782}{arxiv:1608.04782}}} 
\end{tabular} \\ 
\end{tabular}
\end{center}

\begin{center}
\begin{tabular}{m{1.0in}m{0.25in}c}
\raggedleft{\textit{\small{Submitted}}} & & 
\begin{tabular}{m{0.85in}m{0.15in}m{3.75in}}
\multicolumn{3}{p{4.75in}}{\textit{High throughput thermal conductivity of high temperature solid phases: The case of oxide and fluoride perovskites}} \\ \multicolumn{3}{p{4.75in}}{\footnotesize{\textbf{Authors}: Ambroise van Roekeghem, Jes\'{u}s Carrete, \textcolor{NavyBlue}{Corey Oses}, Stefano Curtarolo \& Natalio Mingo \newline \textbf{Abstract}: Using finite-temperature phonon calculations and machine-learning methods, we calculate the mechanical stability of about 400 semiconducting oxides and fluorides with cubic perovskite structures at 0 K, 300 K and 1000 K. We find 92 mechanically stable compounds at high temperatures --- including 36 not mentioned in the literature so far --- for which we calculate the thermal conductivity. We demonstrate that the thermal conductivity is generally smaller in fluorides than in oxides, largely due to a lower ionic charge, and describe simple structural descriptors that are correlated with its magnitude. Furthermore, we show that the thermal conductivities of most cubic perovskites decrease more slowly than the usual $T^{-1}$ behavior. Within this set, we also screen for materials exhibiting negative thermal expansion. Finally, we describe a strategy to accelerate the discovery of mechanically stable compounds at high temperatures. \newline \textbf{arXiv}: \href{http://arxiv.org/abs/1606.03279}{arxiv:1606.03279}}} 
\end{tabular} \\ 
\end{tabular}
\end{center}

\begin{center}
\begin{tabular}{m{1.0in}m{0.25in}c}
\raggedleft{\textit{\small{Submitted}}} & & 
\begin{tabular}{m{0.85in}m{0.15in}m{3.75in}}
\multicolumn{3}{p{4.75in}}{\textit{Combining the {\small AFLOW} {\small GIBBS} and Elastic Libraries for efficiently and robustly screening thermo-mechanical properties of solids}} \\ \multicolumn{3}{p{4.75in}}{\footnotesize{\textbf{Authors}: Cormac H. Toher, \textcolor{NavyBlue}{Corey Oses}, Jose J. Plata, David J. Hicks, Frisco Rose, Ohad Levy, Maarten de Jong, Mark Asta, Marco Fornari, Marco Buongiorno Nardelli \& Stefano Curtarolo \newline \textbf{Abstract}: Thorough characterization of the thermo-mechanical properties of materials requires difficult and time-consuming experiments. This severely limits the availability of data and it is one of the main obstacles for the development of effective accelerated materials design strategies. The rapid screening of new potential systems requires highly integrated, sophisticated and robust computational approaches. We tackled the challenge by surveying more than 3,000 crystalline solids within the {\small AFLOW}\ framework with the newly developed ``Automatic Elasticity Library'' combined with the previously implemented {\small GIBBS}\ method. The first extracts the mechanical properties from automatic self-consistent stress-strain calculations, while the latter employs those mechanical properties to evaluate the thermodynamics within the Debye model. The new thermo-elastic library is benchmarked against a set of 74 experimentally characterized systems to pinpoint a robust computational methodology for the evaluation of bulk and shear moduli, Poisson ratios, Debye temperatures, Gr{\"u}neisen parameters, and thermal conductivities of a wide variety of materials. The effect of different choices of equations of state is examined and the optimum combination of properties for the Leibfried-Schl{\"o}mann prediction of thermal conductivity is identified, leading to improved agreement with experimental results than the {\small GIBBS}-only approach.}} 
\end{tabular} \\ 
\end{tabular}
\end{center}

\begin{center}
\begin{tabular}{m{1.0in}m{0.25in}c}
\raggedleft{\textit{\small{Acta Materialia}}} & & 
\begin{tabular}{m{0.85in}m{0.15in}m{3.75in}}
\multicolumn{3}{p{4.75in}}{\textit{A computational high-throughput search for new ternary superalloys} \newline Acta Mater. \textbf{in press} (2016) } \\ \multicolumn{3}{p{4.75in}}{\footnotesize{\textbf{Authors}: Chandramouli Nyshadham, \textcolor{NavyBlue}{Corey Oses}, Jacob E. Hansen, Ichiro Takeuchi, Stefano Curtarolo \& Gus L. W. Hart \newline \textbf{Abstract}: In 2006, a novel cobalt-based superalloy was discovered with mechanical properties better than some conventional nickel-based superalloys. As with conventional superalloys, its high performance arises from the precipitate-hardening effect of a coherent L1$_{2}$ phase, which is in two-phase equilibrium with the fcc matrix. Inspired by this unexpected discovery of an L1$_{2}$ ternary phase, we performed a first-principles search through 2224 ternary metallic systems for analogous precipitate-hardening phases of the form $X_{3}$[$A_{0.5}, B_{0.5}$], where $X$ = Ni, Co, or Fe, and [$A,B$] = Li, Be, Mg, Al, Si, Ca, Sc, Ti, V, Cr, Mn, Fe, Co, Ni, Cu, Zn Ga, Sr, Y, Zr, Nb, Mo, Tc, Ru, Rh, Pd, Ag, Cd, In, Sn, Sb, Hf, Ta, W, Re, Os, Ir, Pt, Au, Hg, or Tl. We found 102 systems that have a smaller decomposition energy and a lower formation enthalpy than the Co$_{3}$(Al, W) superalloy. They have a stable two-phase equilibrium with the host matrix within the concentration range $0<x<1$ ($X_{3}$[$A_{x}, B_{1-x}$]) and have a relative lattice mismatch with the host matrix of less than or equal to 5\%. These new candidates, narrowed from 2224 systems, suggest possible experimental exploration for identifying new superalloys. Of these 102 systems, 37 are new; they have no reported phase diagrams in standard databases. Based on cost, experimental difficulty, and toxicity, we limit these 37 to a shorter list of six promising candidates of immediate interest. Our calculations are consistent with current experimental literature where data exists. \newline \textbf{arXiv}: \href{http://arxiv.org/abs/1603.05967}{arxiv:1603.05967}}} 
\end{tabular} \\ 
\end{tabular}
\end{center}

\begin{center}
\begin{tabular}{m{1.0in}m{0.25in}c}
\raggedleft{\textit{\small{Chemistry of Materials}}} & & 
\begin{tabular}{m{0.85in}m{0.15in}m{3.75in}}
\multicolumn{3}{p{4.75in}}{\textit{Modeling Off-Stoichiometry Materials with a High-Throughput Ab-Initio Approach} \newline Chem. Mater. \textbf{28}(18), 6484--6492 (2016) } \\ \multicolumn{3}{p{4.75in}}{\footnotesize{\textbf{Authors}: Kesong Yang, \textcolor{NavyBlue}{Corey Oses} \& Stefano Curtarolo \newline \textbf{Abstract}: Predicting material properties of off-stoichiometry systems remains a long-standing and formidable challenge in rational materials design. A proper analysis of such systems by means of a supercell approach requires the exhaustive consideration of all possible superstructures, which can be a time-consuming process. On the contrary, the use of quasirandom approximants, although very computationally effective, implicitly bias the analysis toward disordered states with the lowest site correlations. Here, we propose a novel framework designed specifically to investigate stoichiometrically driven trends of disordered systems (i.e., having partial occupation and/or disorder in the atomic sites). At the heart of the approach is the identification and analysis of unique supercells of a virtually equivalent stoichiometry to the disordered material. We employ Boltzmann statistics to resolve system-wide properties at a high-throughput (HT) level. To maximize efficiency and accessibility, we integrated the method within the automatic HT computational framework \textsc{Aflow}. As proof of concept, we apply our approach to three systems of interest, a zinc chalcogenide (ZnS$_{1-x}$Se$_{x}$), a wide-gap oxide semiconductor (Mg$_{x}$Zn$_{1-x}$O), and an iron alloy (Fe$_{1-x}$Cu$_{x}$), at various stoichiometries. These systems exhibit properties that are highly tunable as a function of composition, characterized by optical bowing and linear ferromagnetic behavior. Not only are these qualities successfully predicted, but additional insight into underlying physical mechanisms is revealed. \newline \textbf{DOI}: \href{http://dx.doi.org/10.1021/acs.chemmater.6b01449}{10.1021/acs.chemmater.6b01449}}} 
\end{tabular} \\ 
\end{tabular}
\end{center}

\begin{center}
\begin{tabular}{m{1.0in}m{0.25in}c}
\raggedleft{\textit{\small{Submitted}}} & & 
\begin{tabular}{m{0.85in}m{0.15in}m{3.75in}}
\multicolumn{3}{p{4.75in}}{\textit{Accelerated discovery of new magnets in the Heusler alloy family}} \\ \multicolumn{3}{p{4.75in}}{\footnotesize{\textbf{Authors}: Stefano Sanvito, \textcolor{NavyBlue}{Corey Oses}, Junkai Xue, Anurag Tiwari, Mario Zic, Thomas Archer, Pelin Tozman, Munuswamy Venkatesan, J. Michael D. Coey \& Stefano Curtarolo \newline \textbf{Abstract}: Magnetic materials underpin modern technologies, ranging from data storage to energy conversion to contact-less sensing. However, the development of a new high-performance magnet is a long and often unpredictable process, and only about two dozen feature in mainstream applications. Here we describe a systematic pathway to the discovery of novel magnetic materials, which demonstrates an unprecedented throughput and discovery speed. Based on an extensive electronic structure library of Heusler alloys containing 236,115 prototypical compounds, we have filtered those alloys displaying magnetic order and established whether they can be fabricated at thermodynamical equilibrium. Specifically, we have carried out a full stability analysis for intermetallic Heusler alloys made only of transition metals. Among the possible 36,540 prototypes, 248 are found thermodynamically stable but only 20 are magnetic. The magnetic ordering temperature, $T_{\mathrm{C}}$, has then been estimated by a regression calibrated on the experimental $T_{\mathrm{C}}$ of about 60 known compounds. As a final validation we have attempted the synthesis of a few of the predicted compounds and produced two new magnets. One, Co$_{2}$MnTi, displays a remarkably high $T_{\mathrm{C}}$ in perfect agreement with the predictions, while the other, Mn$_{2}$PtPd, is an antiferromagnet. Our work paves the way for large-scale design of novel magnetic materials at unprecedented speed.}} 
\end{tabular} \\ 
\end{tabular}
\end{center}

\noindent\hspace{0cm}{\Large2015}

\begin{center}
\begin{tabular}{m{1.0in}m{0.25in}c}
\raggedleft{\textit{\small{Computational Materials Science}}} & & 
\begin{tabular}{m{0.85in}m{0.15in}m{3.75in}}
\multicolumn{3}{p{4.75in}}{\textit{The AFLOW Standard for High-Throughput Materials Science Calculations} \newline Comput. Mater. Sci. \textbf{108A}, 233--238 (2015) } \\ \multicolumn{3}{p{4.75in}}{\footnotesize{\textbf{Authors}: Camilo E. Calderon, Jose J. Plata, Cormac H. Toher, \textcolor{NavyBlue}{Corey Oses}, Ohad Levy, Marco Fornari, Amir Natan, Michael J. Mehl, Gus L. W. Hart, Marco Buongiorno Nardelli \& Stefano Curtarolo \newline \textbf{Abstract}: The Automatic-Flow (AFLOW) standard for the high-throughput construction of materials science electronic structure databases is described. Electronic structure calculations of solid state materials depend on a large number of parameters which must be understood by researchers, and must be reported by originators to ensure reproducibility and enable collaborative database expansion. We therefore describe standard parameter values for k-point grid density, basis set plane wave kinetic energy cut-off, exchange-correlation functionals, pseudopotentials, DFT+U parameters, and convergence criteria used in AFLOW calculations. \noindent\begin{itemize}[leftmargin=*] \item This paper was selected for Editor's Choice \end{itemize} \textbf{DOI}: \href{http://dx.doi.org/10.1016/j.commatsci.2015.07.019}{10.1016/j.commatsci.2015.07.019}}} 
\end{tabular} \\ 
\end{tabular}
\end{center}

\begin{center}
\begin{tabular}{m{1.0in}m{0.25in}c}
\raggedleft{\textit{\small{Chemistry of Materials}}} & & 
\begin{tabular}{m{0.85in}m{0.15in}m{3.75in}}
\multicolumn{3}{p{4.75in}}{\textit{Materials Cartography: Representing and Mining Materials Space Using Structural and Electronic Fingerprints} \newline Chem. Mater. \textbf{27}(3), 735--743 (2015) } \\ \multicolumn{3}{p{4.75in}}{\footnotesize{\textbf{Authors}: Olexandr Isayev, Denis Fourches, Eugene N. Muratov, \textcolor{NavyBlue}{Corey Oses}, Kevin M. Rasch, Alexander Tropsha \& Stefano Curtarolo \newline \textbf{Abstract}: As the proliferation of high-throughput approaches in materials science is increasing the wealth of data in the field, the gap between accumulated-information and derived-knowledge widens. We address the issue of scientific discovery in materials databases by introducing novel analytical approaches based on structural and electronic materials fingerprints. The framework is employed to (i) query large databases of materials using similarity concepts, (ii) map the connectivity of materials space (i.e., as a materials cartograms) for rapidly identifying regions with unique organizations / properties, and (iii) develop predictive Quantitative Materials Structure-Property Relationship models for guiding materials design. In this study, we test these fingerprints by seeking target material properties. As a quantitative example, we model the critical temperatures of known superconductors. Our novel materials fingerprinting and materials cartography approaches contribute to the emerging field of materials informatics by enabling effective computational tools to analyze, visualize, model, and design new materials. \noindent\begin{itemize}[leftmargin=*] \item This paper was selected for Editor's Choice. \end{itemize} \textbf{DOI}: \href{http://dx.doi.org/10.1021/cm503507h}{10.1021/cm503507h}}} 
\end{tabular} \\ 
\end{tabular}
\end{center}
\end{document}